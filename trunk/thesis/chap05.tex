\chapter{Multiresolu��o e Nivel de Detalhe}
\section{Erro Perpendicular}
No caso el�ptico, o problema de determinar o ponto de uma elipse $\mathbf{s}_i$ 
do conjunto que est� sendo agregado ao plano da elipse $\mathbf{\={s}}$ 
que passa a representar esse conjunto n�o � t�o imediato como no caso cicular. Para
calcular a dist�ncia perpendicular entre $\mathbf{s}_i$ e
$\mathbf{\=s}$, temos que empregar o seguinte processo: Seja
$\mathbf{v}(\alpha) = \mathbf{x}_i + \{\mathbf{t}_i^1\cos{\alpha} +
\mathbf{t}_i^2\sin{\alpha}\}$ a equa��o param�trica de $\mathbf{s}_i$,
$\mathbf{\=x}$ o centro de $\mathbf{\=s}$ e $\mathbf{\=n}$ a normal ao plano
$\mathbf{\=H}$ de $\mathbf{\=s}$. 

Para maximizar em $\mathbf{\alpha}$ a dist�ncia entre $\mathbf{v}(\alpha)$ e
$\mathbf{\=H}$ precisamos resolver o pronblema de otimiza��o abaixo:

\begin{eqnarray}
\mathbf{d}_i &=& \mathbf{\max}_{\alpha \in [0,2\pi]}  | <\mathbf{v}(\alpha)
- \mathbf{\=x},\mathbf{\=n} > | \nonumber \\ 
&=& | <\mathbf{x}_i - \mathbf{\=x},\mathbf{n} >  +  \mathbf{\max}_{\alpha \in
 	[0,2\pi]} [ <\mathbf{t}_i^1,\mathbf{n}>\cos{\alpha} +
 	<\mathbf{t}_i^2,\mathbf{n}>\sin{\alpha} ] |
\label{eq1}
\end{eqnarray}





