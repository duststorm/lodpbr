\begin{abstract}
Representa��o digital de superf�cies de modelos reais, podem ser
obtidas automaticamente usando v�rios dispositivos  como $3D$ \textit{scanners}
e fotografica $3D$. Estes dispositvos s�o r�pidos e precisos, o que resulta em
modelos $3D$ com resolu��es de ordem de magnitude elevada. Muitas aplica��es em
computa��o gr�fica podem se beneficiar desse m�todo autom�tico de adquirir
geometria digital incluindo: aplica��es \textit{CAD} (\textit{Computer-Aided
Desing}\abbrev{CAD}{\textit{Computer-AidedDesing}}), simula��o f�sica, realidade
virtual, imagens m�dicas, arquitetura, arqueologia, efeitos especiais, anima��o
e jogos eletr�nicos.

Infelizmente, essa geometria produzida por esse tipo de m�dia vem com um pre�o
de uma grande, talvez gigante, quantidade de dados que requer novas estruturas
de dados e algoritmos para tratar estes objetos.

Visando renderiza��o interativa e com qualidade, modelos produzidos por esse
tipo de m�dia, que n�o s�o nada mais que nuvem de pontos, s�o convertitos em
representa��o em \textit{splats}, que s�o discos ou elipses orientadas.
Esta disserta��o propoe um algoritmo de simplifica��o que trabalha diretamente
com \textit{splats}, suprindo a necessidade de um simplifica��o de pontos puros
e uma posterior convers�o desta nuvem simplificada em representa��o baseada em
\textit{splats}.
\end{abstract}

